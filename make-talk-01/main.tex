\pdfminorversion=4
\documentclass[aspectratio=169]{beamer}

\mode<presentation>
{
  \usetheme{default}
  \usecolortheme{default}
  \usefonttheme{default}
  \setbeamertemplate{navigation symbols}{}
  \setbeamertemplate{caption}[numbered]
  \setbeamertemplate{footline}[frame number]  % or "page number"
  \setbeamercolor{frametitle}{fg=white}
  \setbeamercolor{footline}{fg=black}
}

\usepackage[english]{babel}
\usepackage{inputenc}
\usepackage{tikz}
\usepackage{courier}
\usepackage{array}
\usepackage{bold-extra}
\usepackage{minted}
\usepackage[thicklines]{cancel}
\usepackage{fancyvrb}
\usepackage[normalem]{ulem}

\xdefinecolor{dianablue}{rgb}{0.18,0.24,0.31}
\xdefinecolor{darkblue}{rgb}{0.1,0.1,0.7}
\xdefinecolor{darkgreen}{rgb}{0,0.5,0}
\xdefinecolor{darkgrey}{rgb}{0.35,0.35,0.35}
\xdefinecolor{darkorange}{rgb}{0.8,0.5,0}
\xdefinecolor{darkred}{rgb}{0.7,0,0}
\definecolor{darkgreen}{rgb}{0,0.6,0}
\definecolor{mauve}{rgb}{0.58,0,0.82}

\title[2024-07-08-scipy-teen-track-talk-01]{Welcome to the SciPy 2024 Teen Track}
\author{Jim Pivarski}
\institute{Princeton University -- IRIS-HEP}
\date{July 8, 2024}

\usetikzlibrary{shapes.callouts}

\begin{document}

\logo{\pgfputat{\pgfxy(0.11, 7.4)}{\pgfbox[right,base]{\tikz{\filldraw[fill=dianablue, draw=none] (0 cm, 0 cm) rectangle (50 cm, 1 cm);}\mbox{\hspace{-8 cm}\includegraphics[height=1 cm]{princeton-logo-long.png}\hspace{0.1 cm}\raisebox{0.1 cm}{\includegraphics[height=0.8 cm]{iris-hep-logo-long.png}}\hspace{0.1 cm}}}}}

\begin{frame}
  \titlepage
\end{frame}

\logo{\pgfputat{\pgfxy(0.11, 7.4)}{\pgfbox[right,base]{\tikz{\filldraw[fill=dianablue, draw=none] (0 cm, 0 cm) rectangle (50 cm, 1 cm);}\mbox{\hspace{-8 cm}\includegraphics[height=1 cm]{princeton-logo.png}\hspace{0.1 cm}\raisebox{0.1 cm}{\includegraphics[height=0.8 cm]{iris-hep-logo.png}}\hspace{0.1 cm}}}}}

% Uncomment these lines for an automatically generated outline.
%\begin{frame}{Outline}
%  \tableofcontents
%\end{frame}

% START START START START START START START START START START START START START

\begin{frame}{Welcome!}
\begin{center}
\includegraphics[width=0.5\linewidth]{SCIPY-2024-no-textArtboard+1@3x.png}
\end{center}
\end{frame}

\begin{frame}{What this session is about}
\Large
\vspace{0.5 cm}

The SciPy Teen Track is an introduction to scientific problem-solving with Python.

\vspace{1 cm}
\uncover<2->{That's a broad topic: it could involve any academic science, data science, or business analytics.}

\vspace{1 cm}
\uncover<3->{I decided to focus on machine learning.}
\end{frame}

\begin{frame}{Schedule}
\vspace{0.5 cm}

\large
\textcolor{darkblue}{Monday, July 8, 2024}

\small
\vspace{0.1 cm}
{\bf 9:30am -- 12pm:}

\normalsize
\vspace{0.1 cm}
\begin{itemize}
\item Python review: just enough Python knowledge for the rest of the session
\item Mathematical background: just the concepts
\item Project: fitting a theoretical model to experimental measurements
\end{itemize}

\small
\vspace{0.1 cm}
{\bf 1pm -- 3:30pm:}

\normalsize
\vspace{0.1 cm}
\begin{itemize}
\item Scikit-Learn, a machine learning library
\item From linear fitting to neural networks
\end{itemize}

\large
\vspace{0.25 cm}
\textcolor{darkblue}{Tuesday, July 9, 2024}

\small
\vspace{0.1 cm}
{\bf 1pm -- 3:30pm:}

\normalsize
\vspace{0.1 cm}
\begin{itemize}
\item Big data: the complete works of Shakespeare
\item Project: building a language model
\item Large language models (e.g.\ ChatGPT)
\end{itemize}
\end{frame}

\begin{frame}{Software}
\Large
\vspace{0.5 cm}

We'll be using JupyterLite, a variant of Jupyter that runs in your web browser, so there's nothing to install.

\vspace{1 cm}
\uncover<2->{I'll provide URLs and guide you through using Jupyter as needed.}
\end{frame}

\begin{frame}{Who am I?}
\Large
\vspace{0.5 cm}

Jim Pivarski, a computational physicist at Princeton University.

\normalsize
\vspace{1 cm}
I studied particle physics at CMU and Cornell and helped prepare the Large Hadron Collider at CERN for the Higgs boson discovery.

\vspace{0.5 cm}
I started using Python (version 2.1!) in 2002 and wrote my first package in 2005---for fitting, which is what we'll be doing today. I spent some years as a data scientist in the private sector, but now I'm developing scientific software for particle physicists.
\end{frame}


\end{document}
